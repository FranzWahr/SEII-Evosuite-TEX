\documentclass[pdftex,english,oribibl]{llncs}

%% Spracheinstellungen laden
\usepackage[english]{babel}

%% Schriftart in der Ausgabe/Eingabe
\usepackage[T1]{fontenc}
\usepackage{textcomp}
\usepackage[latin1]{inputenc}

%% Zitate
\usepackage[numbers]{natbib}
\bibliographystyle{abbrvnat}
%\bibliographystyle{dinat}
%\bibliographystyle{plainnat}
%\bibliographystyle{splncs}
%% Similar to option "sectionbib" but \refname instead of \bibname
\makeatletter
\renewcommand\bibsection{\section*{\refname\@mkboth{\MakeUppercase{\refname}}{\MakeUppercase{\refname}}}}
\makeatother

%% Index
%\usepackage{makeidx}
%\makeindex

%% PDF Einstellungen
% muss nach natbib geladen werden!
\usepackage{nameref}
\usepackage{varioref}
\usepackage[pdfusetitle,pdftex,colorlinks]{hyperref}
\hypersetup{pdfborder={0 0 0}}
\hypersetup{bookmarksdepth=3}
\hypersetup{bookmarksopen=true}
\hypersetup{bookmarksopenlevel=1}
\hypersetup{bookmarksnumbered=true}
\usepackage{color}
\hypersetup{colorlinks=false}
\usepackage{titling}

%\usepackage[section]{tocbibind}

\makeatletter
\gdef\@keywords{}
\def\keywords#1{\gdef\@keywords{#1}}
\gdef\@subtitle{}
\def\subtitle#1{\gdef\@subtitle{#1}}

%% modified from llncs
\renewenvironment{abstract}{%
  \list{}{\advance\topsep by0.35cm\relax\small%
          \leftmargin=1cm%
          \labelwidth=\z@%
          \listparindent=\z@%
          \itemindent\listparindent%
          \rightmargin\leftmargin}%
          \item[\hskip\labelsep\bfseries\abstractname]}{%
  \if!\@keywords!\else{\item[~]\item[\hskip\labelsep\bfseries\keywordname]\@keywords}\fi%
  \endlist}

\AtBeginDocument{%
  \if!\@subtitle!\else\hypersetup{pdfsubject={\@subtitle}}\fi
  \if!\@keywords!\else\hypersetup{pdfkeywords={\@keywords}}\fi
}
\makeatother

% llncs hyperref fix
\makeatletter
\providecommand*{\toclevel@author}{0}
\providecommand*{\toclevel@title}{0}
\makeatother

%% Grafiken
\usepackage[pdftex]{graphicx}
\DeclareGraphicsExtensions{.pdf,.jpg,.png}
\usepackage{subfigure}

%% Mathe
\usepackage{amsmath}
\usepackage{amssymb}

%% Listings
\usepackage{listings}
\lstset{escapechar=\%, frame=tb, basicstyle=\footnotesize}

%% Sonstiges
\newcommand{\TODO}[1]{\par\textcolor{red}{#1}\marginpar{\textcolor{red}{TODO}}}
\newcommand{\TODOX}[1]{\textcolor{red}{#1}\marginpar{\textcolor{red}{TODO}}}
\pagestyle{plain}

% Keine "Schusterjungen"
\clubpenalty = 10000
% Keine "Hurenkinder"
\widowpenalty = 10000 \displaywidowpenalty = 10000

%%%%%%%%%%%%%%%%%%%%%%%%%%%%%%%%%%%%%%%%%%%%%%%%%%%%%%%%%%%%%%%%%%%%%%%%%%%%%%%
%%% BEGIN DOCUMENT
%%%%%%%%%%%%%%%%%%%%%%%%%%%%%%%%%%%%%%%%%%%%%%%%%%%%%%%%%%%%%%%%%%%%%%%%%%%%%%%
\pretitle{%
  \begin{center}\LARGE
  \noindent\includegraphics[height=2.5cm]{figures/logo-se}\hfill{}\includegraphics[height=2.5cm]{figures/logo-uni}\\\vspace{0.5cm}
}
\title{Evosuite Report-Paper}
% \subtitle{My (optional) Subtitle}
\author{Hoang, Lam\\ Peverali, Francois \\ Allogie, Jamal}
\institute{Humboldt University of Berlin\\Department of Computer Science\\12489 Berlin, Germany}
\posttitle{\end{center}}


\begin{document}

\maketitle

\begin{abstract}
  Automatically generating test cases for finding defects in program code helps to efficiently build qualitative software. \textsc{Evosuite} is a tool that both helps generating \textit{test cases} and \textit{oracles}, that assert the correctness of a programs behaviour at runtime, but also allows to identify a test suites coverage. Thereby it is using a novel approach that aims at optimizing whole \textit{test suites} by approaching a coverage criterion using a mutation-based assertion generation. The following paper wants to validate the approach by comparing the test suite generated by \textsc{Evosuite}  to a given Java benchmark and its own test suite.
\end{abstract}

\section{Introduction}

This paper describes how EvoSuite works: \citet{fraser2011evosuite}.

This is my introduction. \citet{Shaw2003WritingGoodSoftwareEngineeringResearchPapersMinitutorial} wrote a paper with hints on how to write good software engineering research papers. By the way, this was an example for using the \textit{natbib} command \texttt{\textbackslash{}citet\{\}}.

  Section~\ref{sec:anotherSection} presents everything one must know. The conclusions follow in Section~\ref{sec:conclusions}.

\subsection{Contribution}

Our contributions consist of...(Am Ende der introduction sollt ihr klar (in einer Liste) eure persönlichen contributions darstellen, also prinzipiell den Mehrwert eurer durchgeführten Versuche.)


\section{Background}\label{sec:anotherSection}

Software-testing within software-engineering deals with systematically verifying the functionality of the tested program and thereby assuring its quality. For increasing the efficiency of writing test-cases automated methods that seek to generate test-cases by iterating through a possible search-space are being used. A comprehensive comparison of both a traditional and a novel approach in search-based software testing is offered by Rojas et al. \citep{rojas2017detailed}. A common scenario for software-based software testing consists in generating "a set of test cases maximizing their code coverage or maximizing their fault detection capability" being branch coverage a de facto standard coverage criterion within the reasearch field. Furthermore it states that the traditional approach can be characterized as a single goal approach, i.e. that each coverage goal is approximated individually. Whereas for the novel \textit{whole test suite} approach proposed by Rojas et al. "the problem is changed to a search for a set of tests that covers all coverage goals at the same time", thereby solving two issues of the traditional approach, which are given by the \textit{search budget distribution problem} and the \textit{ordering of the coverage goals}. The former consisting in the fact, that the infeasibility of coverage goals is an undecidable problem, which means that a test generation software could unnecessarily spend ressources on test generation for a coverage goal that could never be reached. The latter ...TODO

A detailed description for the traditional approach is provided by McMinn \citep{mcminn2004search} and Wegener \citep{wegener2001evolutionary},  whereas a more detailed description of the whole test suite approach can be found at Fraser and Arcuri \citep{fraser2015achieving}.

\subsection{Problem Description} 


TODO:
....(Erklärt das behandelte Problem und warum dessen Lösung wichtig ist. Gebt dem Leser ausreichend Hintergrundinformationen um eure Arbeit zu verstehen. Stellt die benutzten/behandelten Tools dar und ggf. eure durchgeführten Modifikationen. Gebt ausreichend exemplarisches Material in Form von figures, tables, etc.)

That's all template stuff:  \textit{AspectJ} can be used to weave cross-cutting concerns into Java programs \citep{AspectJ2007}. By the way, this was an example for using the \textit{natbib} command \texttt{\textbackslash{}citep\{\}}.

  We will now demonstrate how to use subfigures (see Figure~\ref{fig:subfig}). Figure~\ref{fig:circle} shows a circle. A star is displayed in Figure~\ref{fig:star}.

  \begin{figure}
    \centering
    \subfigure[This is a circle.]{\label{fig:circle}
      \includegraphics[width=0.3\textwidth]{figures/template_circle.pdf}
    }
    \subfigure[This is a star.]{\label{fig:star}
      \includegraphics[width=0.3\textwidth]{figures/template_star.pdf}
    }
    \caption{A circle (a) and a star (b). Note that any caption ends with a full stop character.}
    \label{fig:subfig}
  \end{figure}

\section{Evaluation}\label{sec:evaluation}

  These are my evaluations.
  
  \subsection{Solution and Evaluation Description}
  
  ...(Beschreibt die Lösung des behandelten Problems, bzw. eure durchgeführten Analysen. Gebt zudem auch Informationen zu benutzten Metriken, wie diese zu berechnen sind, und über den Aufbau eurer Experimente.)
  
  
\section{Discussion}\label{sec:discussion}

  ...discussion.
  
    \subsection{Results Discussion} 
    ...(Stellt die eure Ergebnisse vor und erklärt den Ausgang eurer Experimente. Zieht Folgerungen aus den Ergebnissen. Beschreibt die nächsten Schritte, die durchgeführt werden müssten/könnten und anderen "Future Work".)
    
    \subsection{Related Work}
    ...Geht kurz auf themenverwandte Arbeiten ein. Was unterscheidet eure Arbeit von anderen Ansätzen, etc.?
    
    \citep{rojas2017detailed}

\section{Conclusions}\label{sec:conclusions}

  These are my conclusions.
  
\bibliography{template}


\end{document}
